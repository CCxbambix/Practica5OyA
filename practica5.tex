\documentclass{article}

%Paquetes
\PassOptionsToPackage{table,dvipsnames,x11names}{xcolor}
\usepackage[spanish]{babel}
\usepackage[hidelinks]{hyperref}
\usepackage{listings}
\usepackage{fancyhdr}
\usepackage{amssymb}
\usepackage{textcomp} 
\usepackage{geometry}
\usepackage{enumitem}
\usepackage{amsmath}
\usepackage{twemojis}
\usepackage{logix}
\usepackage{xcolor}
\usepackage{colortbl}
\usepackage{tikz}
\usepackage{booktabs}
\usepackage{float}
\usetikzlibrary{calc,positioning,matrix}
\pgfdeclarelayer{background}
\pgfsetlayers{background,main}

\lstdefinestyle{mystyle}{
  backgroundcolor=\color{backcolour},   
  commentstyle=\color{codegreen},
  keywordstyle=\color{magenta},
  numberstyle=\tiny\color{codegray},
  stringstyle=\color{codepurple},
  basicstyle=\ttfamily\footnotesize,
  breakatwhitespace=false,         
  breaklines=true,                 
  captionpos=b,                    
  keepspaces=true,                 
  numbers=left,                    
  numbersep=5pt,                  
  showspaces=false,                
  showstringspaces=false,
  showtabs=false,                  
  tabsize=2
}

\lstset{style=mystyle}

%Documento
\begin{document}
  \begin{titlepage}
  	\vspace*{0.8cm}
  	\centering
  	{\bfseries\huge Universidad Nacional Autónoma de México \par}
  	\vspace{1cm}
  	{\scshape\huge Facultad de Ciencias \par}
  	\vspace{2cm}
  	{\scshape\Huge Práctica 05 \par}
  	\vfill
		{\scshape\Large Lenguaje Ensamblador \par}
  	\vspace{1.5cm}
  	{\scshape\Large Organización y Arquitectura de Computadoras \par}
  	\vfill
  	{\scshape\Large Semestre: 2026-1\par}
  	\vspace{1.5cm}
		{\Large Fecha de entrega: 23 de Octubre de 2025\par}
		\vspace{1.5cm}
		{\Large Integrantes del equipo:\par}
		\begin {itemize}
		\item Monroy Flores Alexa Sofía
		\item Salazar Enríquez Luis Alberto
		\end{itemize}
  \end{titlepage}

	\newgeometry{left=1.8cm, right=2cm, top=2.8cm, bottom=2.4cm}
	\pagestyle{fancy}
	\fancyhead[RO,L]{\textbf{\makebox[\textwidth][r]{\ensuremath{\OutlineCurvedDiamond}Práctica 5\ensuremath{\LoopArrowRight}Monroy Flores\ensuremath{\parallel}Salazar Enriquez}}}
	\section*{Preguntas}
  \begin{enumerate}
    \item \textbf{¿Que hacen las instrucciones de tipo FR y de tipo FI? Da algunos ejemplos de instrucciones de este tipo y menciona porque están separadas de las otras 3 principales.\\}
	Las instrucciones de FR son instrucciones las cuales trabajan con datos los cuales ya estan en los registros, pero dichos datos son de punto flotante.
	Las Instrucciones de FI trabajan con datos que son de registros y que son inmediatos, pero tienen la diferencia que también trabajan con datos de punto flotante
	ejemplos de este tipo de instrucciones:
	add.d 
	add.s
	l.s
	l.d
	mul.d
	mul.s
	Porque las otras instrucciones trabajan con números enteros, para este tipo de instrucciones cuentan con un coprocesador aparte dedicado a ellas, las cuales también tienen un formato diferente de 		instrucción.

    \item \textbf{¿Que es la Portabilidad de Arquitecturas (Cross-Architecture Porting)\\}
    \item \textbf{¿Que son las arquitecturas x84 y x64? ¿Tienen un  ́unico lenguaje ensamblador?\\}
    \item \textbf{En el contexto de los lenguajes de ensamblador, ¿Que es NASM? ¿Qué es MASM?\\}
	NASM en un software para programar en ensamblador en una arquitectura de x86 CPU que es de arquitectura portable a plataforma modernas, como tambien generacion de codigo para plataformas nuevas y viejas.
	MASM es en macroensamblador de Microsoft para arquitecturas de 32 y 62 bits, este brinda un mayor control en el hardware y tambien disminuye la sobrecarga del código en tiempo y en memoria.

  \end{enumerate}
  \section*{Referencias:}
		\begin{itemize}
			\item si
		\end{itemize}
	\restoregeometry
\end{document}
